\documentclass{article}

\usepackage{tabularx}
\usepackage{booktabs}

\title{Reflection Report on Artificial Neural Netwrok (ANN)}

\author{Tanya Djavaherpour}

\date{}

\input{../Comments}
\input{../Common}

\begin{document}

\maketitle

This reflection document details the changes made in response to 
feedback on the Software Requirements Specification (SRS), Design 
Document, and Verification and Validation (VnV) Plan for the 
Artificial Neural Network (ANN) system. It describes the iterative 
process and key decisions made for the system, followed by a summary 
of the project’s goals, achievements, and anticipated future modifications.

\section{Changes in Response to Feedback}

This section outlines how comments and suggestions from the instructor and 
teammates were instrumental in achieving success.

\subsection{SRS}

The project was systematically planned using the SRS document, which provided 
a roadmap for ongoing ANN system development. The initial development steps 
and all requirements were documented. Introducing a domain expert and a 
secondary reviewer proved effective for quality feedback. Final SRS 
adjustments were influenced by Dr. Smith's crucial feedback, though 
some of his views diverged from the author's, as detailed in
\href{https://github.com/tanya-jp/ANN-CAS741/issues/4}{this issue link}. 
Changes commits based on Dr. Smith's comments are also accessible through the mentioned link as well.

\subsection{Design and Design Documentation}
The initial drafts of the Module Guide (MG) and Module Interface Specification 
(MIS) significantly aided implementation, despite undergoing many changes. 
Reviewed by domain experts, secondary expert, and Dr. Smith, their usefulness 
was enhanced. Feedback from Dr. Smith and the author's comments can be 
found \href{https://github.com/tanya-jp/ANN-CAS741/issues/21}{here}.

% \subsection{VnV Plan and Report}
\subsection{VnV Plan}
The VnV plan was crucial in defining the requirements. It was updated 
following reviews by experts and Dr. Smith, similarly to the SRS document.
 Details on the changes influenced by Dr. Smith's feedback and the author's 
 responses are available \href{https://github.com/tanya-jp/ANN-CAS741/issues/13}{here}.

\section{Design Iteration (LO11)}

The development documents were subjected to several iterations for updates. 
The first iteration, which included the original document, was developed and verified by authour. 
In the second iteration, areas needing improvement were identified by the domain expert. 
All feedback was systematically collected on GitHub issues for easy reference, 
and these issues were updated instantaneously to prevent their recurrence during 
the third iteration, which was a secondary peer review. 
The final verification of the document was performed by Dr. Smith. 
The use of this process is highly recommended for future software development projects.


\section{Design Decisions (LO12)}


This project involved redeveloping a previous version initially created in a 
Jupyter Notebook without separate modules. Our goal was to modularize the structure 
and improve accuracy. Additionally, we implemented features to save the trained model 
and incorporate an upload function.

In the context of the Artificial Neural Network (ANN), we increased our dataset size by 
2.5 times and initially expected a corresponding decrease in model accuracy, predicting 
it might fall to about 18\% (45 divided by 2.5) of its original level.
Our accuracy was over 20\%. This better-than-anticipated 
result not only confirms the validity of our approach but also 
emphasizes the improved generalization ability of our model, marking 
significant progress towards achieving our project goals.

% \section{Economic Considerations (LO23)}

% \plt{Is there a market for your product? What would be involved in marketing your 
% product? What is your estimate of the cost to produce a version that you could 
% sell?  What would you charge for your product?  How many units would you have to 
% sell to make money? If your product isn't something that would be sold, like an 
% open source project, how would you go about attracting users?  How many potential 
% users currently exist?}

\section{Reflection on Project Management (LO24)}

This section focuses on processes and tools used for project management.

\subsection{How Does Your Project Management Compare to Your Development Plan}

Everything proceeded as planned across all the steps.

% \subsection{What Went Well?}

% \plt{What went well for your project management in terms of processes and 
% technology?}

\subsection{What Went Wrong?}

Despite knowing that achieving very high accuracy with ANNs is 
challenging, I had set my expectations higher than what was realistically 
achievable. Additionally, I initially updated the design documents to 
reflect changes made during my implementation, based on the feedback 
received from my presentation. Only after this did I review the 
feedback from the reviewers and Dr. Smith. Looking back, it would 
have been more effective to review all the feedback first before making 
any modifications.

\subsection{What Would you Do Differently Next Time?}

Next time, I would select a project for which there is clear evidence 
of successful outcomes. This approach would likely enhance the project's 
feasibility and increase the chances of achieving high-quality results.

% \section{Reflection on Capstone}

% \plt{This question focuses on what you learned during the course of the capstone project.}

% \subsection{Which Courses Were Relevant}

% \plt{Which of the courses you have taken were relevant for the capstone project?}

% \subsection{Knowledge/Skills Outside of Courses}

% \plt{What skills/knowledge did you need to acquire for your capstone project
% that was outside of the courses you took?}

\end{document}