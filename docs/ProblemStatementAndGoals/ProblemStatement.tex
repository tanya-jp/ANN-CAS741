\documentclass{article}

\usepackage{tabularx}
\usepackage{booktabs}
\usepackage{paralist}

\title{Problem Statement and Goals\\Artificial Neural Network}

\author{Tanya Djavaherpour}

\date{January 19, 2024}

%% Comments

\usepackage{color}

\newif\ifcomments\commentstrue %displays comments
%\newif\ifcomments\commentsfalse %so that comments do not display

\ifcomments
\newcommand{\authornote}[3]{\textcolor{#1}{[#3 ---#2]}}
\newcommand{\todo}[1]{\textcolor{red}{[TODO: #1]}}
\else
\newcommand{\authornote}[3]{}
\newcommand{\todo}[1]{}
\fi

\newcommand{\wss}[1]{\authornote{blue}{SS}{#1}} 
\newcommand{\plt}[1]{\authornote{magenta}{TPLT}{#1}} %For explanation of the template
\newcommand{\an}[1]{\authornote{cyan}{Author}{#1}}

%% Common Parts

\newcommand{\progname}{ProgName} % PUT YOUR PROGRAM NAME HERE
\newcommand{\authname}{Team \#, Team Name
\\ Student 1 name
\\ Student 2 name
\\ Student 3 name
\\ Student 4 name} % AUTHOR NAMES                  

\usepackage{hyperref}
    \hypersetup{colorlinks=true, linkcolor=blue, citecolor=blue, filecolor=blue,
                urlcolor=blue, unicode=false}
    \urlstyle{same}
                                


\begin{document}

\maketitle

\begin{table}[hp]
\caption{Revision History} \label{TblRevisionHistory}
\begin{tabularx}{\textwidth}{llX}
\toprule
\textbf{Date} & \textbf{Developer(s)} & \textbf{Change}\\
\midrule
January 19, 2024 & Tanya Djavaherpour & Initial Draft\\
January 26, 2024 & Tanya Djavaherpour & Fixx Problem Statement\\
% ... & ... & ...\\
\bottomrule
\end{tabularx}
\end{table}

\section{Problem Statement}

In the rapidly evolving field of artificial intelligence and machine learning, image classification 
stands as a cornerstone application, enabling machines to interpret and analyze visual data with 
increasing accuracy. This project addresses the specific challenge of image classification using 
Artificial Neural Networks (ANN) on the CIFAR-10 dataset, a well-known benchmark in the machine learning community.

\subsection{Problem}

CIFAR-10 is divided into five training batches and one test batch, 
each with 10000 images. The test batch contains exactly 1000 randomly-selected 
images from each class. The training batches contain the remaining images in random order, 
but some training batches may contain more images from one class than another. Between them, 
the training batches contain exactly 5000 images from each class.

In a preceding project, available on \href{https://github.com/tanya-jp/CIFAR-Classification/tree/main}{GitHub}, an ANN-based model was developed to classify images 
from a reduced subset of the CIFAR-10 dataset, encompassing only 4 out of the 10 available classes. 
This reduction was primarily to manage computational complexity. However, the model achieved a modest 
accuracy of less than 50\%, indicating substantial room for improvement.

The current project aims to extend this prior work by incorporating all 10 classes of the 
CIFAR-10 dataset, thus significantly increasing the scope and challenge of the classification task. 
The primary objective is to improve the accuracy of the image classification model by modifying the 
ANN architecture. This involves experimenting with various configurations, such as adjusting the number 
of neurons and layers, to find an optimal structure that enhances performance.

Additionally, a novel feature is planned for this project: the ability to accept an image input 
(via directory input) from the user, process it through the trained model, and output the classified category. 


\subsection{Inputs and Outputs}

Inputs: The complete CIFAR-10 dataset, encompassing 10 categories of images, each representing different 
objects like animals and vehicles. User-provided images for classification.

Outputs: Classification accuracy of the ANN model, measured against a benchmark. For user-provided images, 
the output will be the category name into which the image is classified.

% \subsection{Stakeholders}

\subsection{Environment}

The software is compatible with various types of operating systems such as Windows, 
Linux, or macOS and should work on various types of personal computers and laptops.

% \wss{Hardware and software}

\section{Goals}

The project aims to achieve the following goals:

\begin{inparaitem}  
\item To enhance the accuracy and performance of the Artificial Neural Network model for classifying 
images in the CIFAR-10 dataset.

\item To explore and refine the neural network architecture, balancing computational efficiency with 
classification accuracy.

\item To develop a user-friendly terminal-based interface that allows users to interact with the 
model and test its performance.

\item To produce comprehensive and accessible documentation detailing the development process, 
model architecture, and implementation techniques.
\end{inparaitem}

\section{Stretch Goals}

A key stretch goal for this project is to implement a user interface that is easy for everyone to use, 
both online and offline. This interface would enable users to upload images in any format. Additionally, 
by utilizing a dataset like CIFAR-100 for training the model, we could classify a wider range of image classes. 
However, it's important to note that using a larger dataset like this would significantly increase computational complexity.

\end{document}